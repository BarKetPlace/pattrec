\documentclass[a4paper]{report}

\usepackage[utf8]{inputenc} %Accent
%\usepackage{libertine} %Font
\usepackage[english, francais]{babel} %langue

\usepackage{graphicx} %Include fig
\usepackage{caption} %center the caption
\usepackage{subfig} %Include subfig
\usepackage{lastpage} %ref LastPage 
\usepackage{fancyhdr} % headers,footers
\usepackage{multicol} % minipages
\usepackage{textcomp} 
\usepackage{lscape}   %Format paysage
\usepackage{fancybox} %Image arrière plan
\usepackage{amsmath} %\mathbb, \mathit...
\usepackage{amssymb} 
\usepackage{color} %couleurs
\usepackage{float}
\usepackage[hidelinks]{hyperref} %Liens intradoc et url
\usepackage{titlepic}

%\usepackage{algorithm}
%\usepackage{algorithmic} %Algo en pseudo code
%\usepackage{algorithm2e} %for psuedo code

%\usepackage{boxedminipage} %Surligner

%\newcounter{apppage} % Annexes

%Dossier contenant les figures
\graphicspath{{figures/}}

%Mise en page
\voffset -1.5 cm
\textheight 24.3 cm
\topmargin 0 cm
\headheight 0 cm
%\headsep 0.6 cm
\textwidth 16.5 cm
\evensidemargin 0 cm
\marginparsep 0 cm
\marginparwidth 0 cm
\oddsidemargin -.5 cm

%Titre
\title{Pattern recognition\\Report\\Assignment n°1}
\author{Audrey Brouard \and Antoine Honoré}



%Type de numérotation des sections & sous-sections
\renewcommand{\thesection}{\Roman{section}}
\renewcommand{\thesubsection}{\thesection.\arabic{subsection}}

%\renewcommand\thesubfigure{(\alph{subfigure})}
\setlength{\parindent}{0cm}
\setlength{\parskip}{1ex plus 0.5ex minus 0.2ex}
\newcommand{\hsp}{\hspace{20pt}}
\newcommand{\HRule}{\rule{\linewidth}{0.5mm}}

%email
\newcommand{\email}[1]{\href{mailto:#1}{\color{blue} \textsf{#1}}}

%Bibliography
\bibliographystyle{apalike}

%Environnement insersion image
\newcommand{\img}[3]{\begin{figure}[!h] \centering \includegraphics[scale=#2]{#1}\captionsetup{justification=centering} \caption{#3} \label{#1} \end{figure}}
  % commande \img{nom image}{scale}{legende}

%TODO
\newcommand{\todo}[1]{{ \Large \textbf{ \colorbox{yellow}{\color{blue} TODO:}}~#1}}

%pushright
\newenvironment{pushright}[1]{\textbf{#1}
\begin{itemize}\item[\hspace{12pt}]}{\end{itemize}
}

%%%%%%%%%%%%%%%%%%%%%%%%%%%%%%%%%%%%%%%%%%%%%%%%%%%%%%%%%%%%%%
%%%%%%%%%%%%%%%%%%%%%%%%%%%%%%%%%%%%%%%%%%%%%%%%%%%%%%%%%%%%%%
%%%%%%%%%%%%%%%%%%%%%%%%%%%%%%%%%%%%%%%%%%%%%%%%%%%%%%%%%%%%%%
\pagestyle{fancy}  % Activation en-tête et pied de page

%En-tête
\fancyhead[L]{Audrey Brouard \& Antoine Honoré - Pattern recognition - Assignment n°1}
%\fancyhead[C]{}
%\fancyhead[R]{}
% Pied de page
\newcommand{\width}{3cm}
%\fancyfoot[L]{ \includegraphics[width=\width]{logo-gipsa} }
\fancyfoot[C]{ \thepage~/~\pageref{LastPage} }
%\fancyfoot[R]{ \includegraphics[width=\width]{logo-phelma} }

\titlepic{\includegraphics[scale=0.6]{kth-logo}}

\begin{document}

%%%%%%%%%%%%%%%% TITLE %%%%%%%%%%%%%%%%
\begin{titlepage}
  \begin{sffamily}
    \begin{center}

      \textsc{ }\\[1.5cm]

      % % Title
      % \vspace{3cm}
      \HRule \\[0.4cm]
      { \Huge \bfseries Pattern recognition system\\Exercise Project\\[0.4cm] }
      \HRule \\[2.5cm]
      \textsc{\LARGE Report Assignment n°1}~\\[2.5cm]
% Authors
      \begin{minipage}{0.4\textwidth}
        \begin{flushleft} \large
          \emph{\textbf{Etudiant}}\\
          Antoine \textsc{Honoré}\\
          \email{honore@kth.se}
          ~\\~\\~\\
        \end{flushleft}
      \end{minipage}
      \hfill
      \begin{minipage}{0.4\textwidth}
        \begin{flushright} \large
          \emph{\textbf{Etudiant}}\\
          Audrey \textsc{Brouard}\\
          \email{brouard@kth.se}
        \end{flushright}
      \end{minipage}
      \includegraphics[scale=0.5]{kth-logo}

     

      \vfill

      % Bottom of the page
      {\large 2015 Period 1}

    \end{center}
  \end{sffamily}

\end{titlepage}


%%% Local Variables:
%%% TeX-master: "master"
%%% End:
%%%%%%%%%%%%%%%%%%%%%%%%%%%%%%%%%%%%%%%%%%%%%%%%%%%%%%%%%%%%%%%%
%%%%%%%%%%%%%%%%%%%%%%%%%%%%%%%%%%%%%%%%%%%%%%%%%%%%%%%%%%%%%%%%
%%%%%%%%%%%%%%%%%%%%%%%%%%%%%%%%%%%%%%%%%%%%%%%%%%%%%%%%%%%%%%%%
%%%%%%%%%%%%%%%%%%%%%%%%%%%%%%%%%%%%%%%%%%%%%%%%%%%%%%%%%%%%%%%%
%%%%%%%%%%%%%%%%%%%%%%%%%%%%%%%%%%%%%%%%%%%%%%%%%%%%%%%%%%%%%%%%
\section{}

Let us calculate $P(S_{t}=j)$ $ \forall j\in\{1,2\} $ and for t = 1,2,3,...

We will do it $ \forall t\in\{1..3\} $ and notice that this is constant $\forall t$.

\begin{itemize}
  \item For t = 1, the probabilities are given by the matrix $q_{j}$. Thus, $P(S_{1}=1) = 0.75$ and $P(S_{1}=2) = 0.25$.\\
\end{itemize}
$\forall t\geq 2$, we will use the following formula :

	\begin{itemize}
	  \item $P(S_{t}=j) = \sum_{i=1}^{2} P({S_t}=j,S_{t-1}=i) = \sum_{i=1}^{2} P(S_t=j|S_{t-1}=i)P(S_{t-1}=i)$\\
	\end{itemize}

For the case t = 2, we have :
\begin{itemize}
  \item $P(S_t=j) = \sum_{i=1}^{2} a_{ij} q_i$
\end{itemize}

We thus obtain $P(S_{2}=1)=0.75$  and $P(S_{2}=2)=0.25$.
We immediately notice that \[\forall i~P(S_{2}=i) = P(S_{1}=i)\].
By recurrency, we have $\forall t~\Pr(S_{t}=j)$ constant.

\section{}
After we generated 10 000 state integers, we found the following probabilities :
\[\Pr(S_{t} = 1) = 0.7525~and~\Pr(S_{t} = 2) = 0.2475\]

\section{@HMM/rand}
\subsection{Theorical calculation}
Let us now calculate $E[X_{t}]$ and $Var[X_{t}]$.

\begin{itemize}
	\item 
	$E[X_{t}]$ : the book gives the formula $E[X]=E_{S}[E_{X}[X|S]]$, and according to the two different possible values of S, X density probability function is either $b_{1}$ or $b_{2}$.
	
	Then, for j = 1, $E[X|S]=\mu_{1}$ ;
	and for j = 2, $E[X|S]=\mu_{2}$.
	
	
	$E[X]=P(S=1)*\mu_{1} + P(S=2)*\mu_{2} = 0 + 3*0.25 = 0.75$.
	
\end{itemize}

\begin{itemize}
	\item $Var[X_{t}]$ : according to the book, $Var[X_{t}]=E_{S}[Var_{X}[X|S]] + Var_{S}[E_{X}[X|S]]]$.
	
	Thanks to the same observation as before, the expression becomes :
	
	$Var[X_{t}]=[0.75 * \sigma_{1} + 0.25*\sigma_{2}] + [0.75*\mu_{1}^2+0.25*\mu_{2}^2 - 0.75^2] = 1+0.25*9-0.75^2=2.6875$.
\end{itemize}
\pagebreak
\subsection{Measures}
For these measures, we use the section ``Test @HMM/rand'' in the file main.m
As we can see on the figures \ref{hmmrand:var} and \ref{hmmrand:mean} the experimental values are close from the theoretical ones.
\begin{figure}[!h]
\centering
    \subfloat[$Var(X) \simeq 3.316$]{\includegraphics[scale = 0.25]{var_hmmrand}\label{hmmrand:var}}\\
    \subfloat[$E(X) \simeq 0.6981$]{\includegraphics[scale = 0.25]{mean_hmmrand}\label{hmmrand:mean}}
\captionsetup{justification=centering}
    \caption{Validation of @HMM/rand\\\color{blue}{Blue : Plot of the 20 attempts}\\\color{red}{Red : Mean over the 20 attempts}}
\end{figure}
\pagebreak
\section{HMM behavior}
For this part, we changed the behavior of the @HMM/rand function, in order to get a vector from $b_{1}$ (not a scalar like the previous question). So here, each sample is a vector $x_{t} \sim N(\mu_j,\sigma_j^2)$ (where j=1 or 2).

The following figure shows the vector ploted for different values of t.
\img{etude_hmmrand}{0.4}{{\color{blue}Blue : Sample at a time $t_1$. $\mu=-0.0065$ and $\sigma^2=0.985$}\\ {\color{red}Red: Sample at a time $t_2$. $\mu=2.9759$ and $\sigma^2=3.8953$}\\ The width lines represent the mean of the samples}\\
\begin{pushright}{Note}
  It is easy to find out in which state the system was when the sample was produced. On the figure \ref{etude_hmmrand} the samples have very different means (0 et 3) et very different variance.
\end{pushright}
\end{document}
