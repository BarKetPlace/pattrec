\documentclass[a4paper]{report}

\usepackage[utf8]{inputenc} %Accent
%\usepackage{libertine} %Font
\usepackage[english, francais]{babel} %langue

\usepackage{graphicx} %Include fig
\usepackage{caption} %center the caption
\usepackage{subfig} %Include subfig
\usepackage{lastpage} %ref LastPage 
\usepackage{fancyhdr} % headers,footers
\usepackage{multicol} % minipages
\usepackage{textcomp} 
\usepackage{lscape}   %Format paysage
\usepackage{fancybox} %Image arrière plan
\usepackage{amsmath} %\mathbb, \mathit...
\usepackage{amssymb} 
\usepackage{color} %couleurs
\usepackage{float}
\usepackage[hidelinks]{hyperref} %Liens intradoc et url
\usepackage{titlepic}

%\usepackage{algorithm}
%\usepackage{algorithmic} %Algo en pseudo code
%\usepackage{algorithm2e} %for psuedo code

%\usepackage{boxedminipage} %Surligner

%\newcounter{apppage} % Annexes

%Dossier contenant les figures
\graphicspath{{figures/}}

%Mise en page
\voffset -1.5 cm
\textheight 24.3 cm
\topmargin 0 cm
\headheight 0 cm
%\headsep 0.6 cm
\textwidth 16.5 cm
\evensidemargin 0 cm
\marginparsep 0 cm
\marginparwidth 0 cm
\oddsidemargin -.5 cm

%Titre
\title{Pattern recognition\\Report\\Assignment n°1}
\author{Audrey Brouard \and Antoine Honoré}



%Type de numérotation des sections & sous-sections
\renewcommand{\thesection}{\Roman{section}}
\renewcommand{\thesubsection}{\thesection.\arabic{subsection}}

%\renewcommand\thesubfigure{(\alph{subfigure})}
\setlength{\parindent}{0cm}
\setlength{\parskip}{1ex plus 0.5ex minus 0.2ex}
\newcommand{\hsp}{\hspace{20pt}}
\newcommand{\HRule}{\rule{\linewidth}{0.5mm}}

%email
\newcommand{\email}[1]{\href{mailto:#1}{\color{blue} \textsf{#1}}}

%Bibliography
\bibliographystyle{apalike}

%Environnement insersion image
\newcommand{\img}[3]{\begin{figure}[!h] \centering \includegraphics[scale=#2]{#1}\captionsetup{justification=centering} \caption{#3} \label{#1} \end{figure}}
  % commande \img{nom image}{scale}{legende}

%TODO
\newcommand{\todo}[1]{{ \Large \textbf{ \colorbox{yellow}{\color{blue} TODO:}}~#1}}

%pushright
\newenvironment{pushright}[1]{\textbf{#1}
\begin{itemize}\item[\hspace{12pt}]}{\end{itemize}
}

%%%%%%%%%%%%%%%%%%%%%%%%%%%%%%%%%%%%%%%%%%%%%%%%%%%%%%%%%%%%%%
%%%%%%%%%%%%%%%%%%%%%%%%%%%%%%%%%%%%%%%%%%%%%%%%%%%%%%%%%%%%%%
%%%%%%%%%%%%%%%%%%%%%%%%%%%%%%%%%%%%%%%%%%%%%%%%%%%%%%%%%%%%%%
\pagestyle{fancy}  % Activation en-tête et pied de page

%En-tête
\fancyhead[L]{Audrey Brouard \& Antoine Honoré - Pattern recognition - Assignment n°2}
%\fancyhead[C]{}
%\fancyhead[R]{}
% Pied de page
\newcommand{\width}{3cm}
%\fancyfoot[L]{ \includegraphics[width=\width]{logo-gipsa} }
\fancyfoot[C]{ \thepage~/~\pageref{LastPage} }
%\fancyfoot[R]{ \includegraphics[width=\width]{logo-phelma} }

\titlepic{\includegraphics[scale=0.6]{kth-logo}}

\begin{document}

%%%%%%%%%%%%%%%% TITLE %%%%%%%%%%%%%%%%
\begin{titlepage}
  \begin{sffamily}
    \begin{center}

      \textsc{ }\\[1.5cm]

      % % Title
      % \vspace{3cm}
      \HRule \\[0.4cm]
      { \Huge \bfseries Pattern recognition system\\Exercise Project\\[0.4cm] }
      \HRule \\[2.5cm]
      \textsc{\LARGE Report Assignment n°1}~\\[2.5cm]
% Authors
      \begin{minipage}{0.4\textwidth}
        \begin{flushleft} \large
          \emph{\textbf{Etudiant}}\\
          Antoine \textsc{Honoré}\\
          \email{honore@kth.se}
          ~\\~\\~\\
        \end{flushleft}
      \end{minipage}
      \hfill
      \begin{minipage}{0.4\textwidth}
        \begin{flushright} \large
          \emph{\textbf{Etudiant}}\\
          Audrey \textsc{Brouard}\\
          \email{brouard@kth.se}
        \end{flushright}
      \end{minipage}
      \includegraphics[scale=0.5]{kth-logo}

     

      \vfill

      % Bottom of the page
      {\large 2015 Period 1}

    \end{center}
  \end{sffamily}

\end{titlepage}


%%% Local Variables:
%%% TeX-master: "master"
%%% End:
%%%%%%%%%%%%%%%%%%%%%%%%%%%%%%%%%%%%%%%%%%%%%%%%%%%%%%%%%%%%%%%%
%%%%%%%%%%%%%%%%%%%%%%%%%%%%%%%%%%%%%%%%%%%%%%%%%%%%%%%%%%%%%%%%
%%%%%%%%%%%%%%%%%%%%%%%%%%%%%%%%%%%%%%%%%%%%%%%%%%%%%%%%%%%%%%%%
%%%%%%%%%%%%%%%%%%%%%%%%%%%%%%%%%%%%%%%%%%%%%%%%%%%%%%%%%%%%%%%%
%%%%%%%%%%%%%%%%%%%%%%%%%%%%%%%%%%%%%%%%%%%%%%%%%%%%%%%%%%%%%%%%
\section{}


1. Below are represented the pitch and intensity profiles of the 3 songs provided. 
\img{Pitch_intens}{1}{Pitches and Intensity for the 3 melodies}

On the pitch profile we can notice that that's we hardly see the different "useful" pitches as the noise has a pitch of around 1000Hz. If we zoom in a little we can make appear more clearly the different amplitude corresponding to different notes :

\img{Pitch_intens_zoom}{1}{Zoom}

Here on the pitch profile we can distinguish different value of pitch between 100Hz and 300Hz, which corresponds to different notes, globally spread over 1.5 octave.


2. In order to fulfil all the conditions listed in the subject, we desingned our feature extractor as follows :

$\bullet$ First we extract the actual pitches by removing the noise around 1000Hz so that we only have the pitches corresponding to the melody.

$\bullet$ Then we interpolate the vector thus obtained with a vector containing all the notes in the range 20Hz-20kHz. This last vector was built thanks to the constant ratio between 2 consecutive semitons. This way we obtain a vector containing the different notes played in the song, but so far we are dependant of the octave.

$\bullet$ To that purpose, we set a reference to remove the offset of the melody. The reference is defined as the first pitch which is not a silence. After that, we are able to determine of how many semitons above or below the reference the next pitches are located. We thus obtain a vector containing the variation of semiton against the reference for each pitch, and infinity when it's a silence.

$\bullet$ Finally, the duration of each note/silence will be determined by the number of time the source remains in the smae state. To avoid problem of tempo, we will define a vector containing the ratio between to notes durations.


\end{document}
